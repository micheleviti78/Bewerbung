\documentclass[ebner,paper=a4,fontsize=11pt,ngerman,BCOR=10mm]{scrlttr2}% 

\KOMAoptions{foldmarks=false,backaddress=false,parskip=full}

\usepackage[ngerman]{babel}
\usepackage[T1]{fontenc}     
\usepackage[latin9]{inputenc}

\usepackage{marvosym}    

\usepackage{url}

\firstfoot{}

\begin{document}
\pagestyle{empty}

\begin{letter}{Bundesanstalt f\"ur Wasserbau\\
Wedeler Landstra{\ss}e 157\\ 
22607 Hamburg}

\setkomavar{date}{\today}
\setkomavar{subject}{Bewerbung als Wissenschaftlicher Mitarbeiter} 

\setlength{\parindent}{15pt}

\opening{Sehr geehrte Damen und Herren,} 

mein Name ist Michele Viti. Neulich habe
ich das Stellenangebot der Bundesanstalt f{\"u}r Wasserbau mit gro{\ss}em
Interesse gelesen und m{\"o}chte mich darauf bewerben. \\ 
% W\"ahrend meiner
% Berufserfahrung habe ich mich mit Datenprozessierung, Datenauswertung
% und der entsprechenden Softwareentwicklung besch{\"a}ftigt.
% Au{\ss}erdem habe ich an zahlreichen Meetings, Workshops und
% internationalen Konferenzen teilgenommen bei denen ich Daten und
% Resultate pr\"asentiert habe. In diesem Anschreiben werde ich kurz die
% wichtigen Punkte meines Studiums und meiner Berufserfahrung
% zusammenfassen.
% 
% %Software entwicklung real time system expertiment
% 
% %\"Uber die StepStone Webseite habe ich von Ihrem Stellenangebot
% %erfahren und m\"ochte mich darauf bewerben.
% 
\indent An der Perugia Universit\"at in Italien habe ich Physik studiert. Im
Anschluss daran habe ich in Deutschland an der Humboldt Universit\"at
zu Berlin im Fachbereich Teilchen-/Beschleunigerphysik promoviert. In
meinen Studien handelte es sich um Simulation und Test wichtiger
Komponenten von Teilchenbeschleunigern und Detektoren.\\
\indent Nach dem erfolgreichen Abschluss meiner Promotion habe ich in Januar
2010 eine Stelle als wissenschaftlicher Mitarbeiter am Forschungsinstitut DESY
in Hamburg angenommen. In DESY Hamburg habe ich in verschiedenen Gruppen
gearbeitet, n{\"a}mlich die ATLAS Experiment Gruppe, die Strahlendetektor Gruppe
und die Maschine Strahlkontrollen (MSK) Gruppe wo ich momentan t{\"a}tig bin. \\
Innerhalb des ATLAS Experiments habe ich am Projekt ALPHA teilgenommen wo
ich die "`Data Preparation"' Gruppe geleitet habe, die f{\"u}r die Aufbereitung
von gro{\ss}en Mengen von Daten zust{\"a}ndig und an der physikalischen
Datenanalyse beteiligt war.\\
In the Strahlendetektor Gruppe habe ich das Softwareframeworks f\"ur die
Datenverarbeitung und Datenauswertung und die Software f{\"u}r das Data
Acquisition System f\"ur ein neuen bildgebenden CMOS Detektor entwickelt.\\
In der aktuellen Gruppe, die MSK Gruppe, bin ich zust{\"a}ndig f{\"u}r die
Software f{\"u}r Regelsysteme und Datenerfassung von Messger{\"a}ten an
Teilchenbeschleunigern und f{\"u}r die Kontrolle von Schrittmotoren.\\
\indent Bei DESY engagiere ich mich mit komplexen Problemen in der modernen
Grundlagenforschung und Entwicklung, wo eine l{\"o}sungsorientierte Denkweise
und Selbstst\"andigkeit erfordert sind. Meine T{\"a}tigkeiten in DESY haben sich
im Laufe der Jahre von rein physikalischen Aufgaben zu mehr technischen
entwickelt. Deswegen kann ich eine exzellente Kombination von Physikkenntnissen
und Erfahrung mit professionellen Methoden in der Softwareentwicklung
anbieten.\\ 
\indent Alle diese Qualit{\"a}ten entsprechen sehr gut den Anforderungen
dieser Stellte als wissenschaftlicher Mitarbeiter. \"Uber eine Einladung zu
einem pers\"onlichen Vorstellungsgespr\"ach w\"urde ich mich sehr freuen.

% 
% Nach dem erfolgreichen Abschluss meiner Promotion habe ich eine Stelle
% als Post-Doktorand am Forschungsinstitut DESY in Hamburg
% angenommen. Hier arbeitete ich f\"ur das ALFA-Projekt des ATLAS
% Experiments am CERN in Genf. Innerhalb des Projekts habe ich mich mit
% Detektorstrahltestmessung besch\"aftigt. In dem Projekt war ich der
% Koordinator der "`Data Preparation"' Gruppe, die f\"ur die Aufbereitung
% von gro{\ss}en Mengen von Daten zust\"andig und an der physikalischen
% Datenanalyse beteiligt war. Au{\ss}erdem geh{\"o}rte ich zu dem
% Softwareentwicklungsteam des ATLAS Experiments.
% 
% Im Juni 2012 wechselte ich zu der Strahlendetektor-Gruppe in DESY. In
% der Gruppe bin ich zust\"andig f\"ur die Entwicklung des
% Softwareframeworks f\"ur Datenaufbereitung, Daten- und
% Bildverarbeitung und der Software f\"ur das Data Acquisition
% System. Au{\ss}erdem beteilige ich mich an der Auswertung der
% Testdaten.
% 
% Meine Beitr\"age zu diesen Studien umfassen einen breiten Bereich,
% insbesondere Computersimulation, Datenaufnahme, Datenaufbereitung und
% Datenauswertung. Um diese Aufgaben zu erf{\"u}llen, habe ich
% erweiterte statistische Analyseverfahren verwendet, fundierte
% Programmierkenntnisse, insbesondere objektorientierte Programmierung,
% erworben, komplexe Software Frameworks benutzt und Algorithmen
% erzeugt.
% 
% %%f\"ur die Entwicklung neuer
% %%Komponenten und Detektoren, Planung und Teilnahme an Tests f\"ur
% %%Protoypen und Datenanalysen. Details \"uber diese Aufgaben k\"onnen
% %%meinem Lebenslauf entnommen werden.
% 
% Ich habe eine hohe Flexibilit\"at, Selbstst\"andigkeit und
% Belastbarkeit entwickelt, um mich erfolgreich verschiedenen Themen und
% neuen Herausforderungen zu stellen.Seit mehr als 8 Jahren arbeite ich
% innerhalb internationaler Kollaborationen mit verschiedenen
% weltber\"uhmten Instituten und Universit{\"a}ten, wie z.B. der
% Stanford Universit{\"a}t in der USA, dem Forschungszentrum CERN in Genf
% und dem Budka Institut in Novosibirsk, in denen sehr gute
% Kommunikationsf\"ahigkeiten in einer multikulturellen Umgebung und
% teamorientierte Arbeit erfordert sind.
% 
% %Seit ca. 8 Jahren arbeite ich in
% %gro{\ss}en internationalen Kollaborationen, .
% 
% Aufgrund meiner oben genannten T\"atigkeiten entsprechen meine
% Qualit\"aten und Kompetenzen sehr gut den Anforderungen dieser Stelle
% als Wissenschaftlicher Mitarbeiter. \"Uber eine Einladung zu einem pers\"onlichen
% Vorstellungsgespr\"ach w\"urde ich mich sehr freuen.

\closing{Mit freundlichem Gru\ss}
\enlargethispage{6\baselineskip}

\end{letter}
\end{document}

\endinput
