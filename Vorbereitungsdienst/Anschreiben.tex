\documentclass[ebner,paper=a4,fontsize=11pt,ngerman,BCOR=10mm]{scrlttr2}% 

\KOMAoptions{foldmarks=false,backaddress=false,parskip=full}

\usepackage[ngerman]{babel}
\usepackage[T1]{fontenc}     
\usepackage[latin9]{inputenc}

\usepackage{marvosym}    

\usepackage{url}

\firstfoot{}

\begin{document}
\pagestyle{empty}

\begin{letter}{ Beh{\"o}rde f{\"u}r Schule und Berufsbildung\\
Hamburger Stra{\ss}e 31\\ 
22083 Hamburg}

\setkomavar{date}{\today}
\setkomavar{subject}{Bewerbung f{\"u}r den Vorbereitungsdienst} 

\opening{Sehr geehrte Damen und Herren,} 

ich m{\"o}chte mich f{\"u}r den Vorbereitungsdienst zum Quereinstieg in den
Lehrerberuf bewerben.
%{\"U}ber die Webseite der Stadt Hamburg habe
%ch das Stellenangebot mit gro{\ss}em Interesse gelesen und m{\"o}chte mich darauf bewerben.

Meine Leidenschaft ist schon immer die Physik,  Mathematik und  Informatik
gewesen. Seit meiner Kindheit war mir klar, dass ich Physik studieren wollte.
Mit acht Jahren habe ich mein erstes Computerprogramm geschrieben, nachdem mein
Vater mich in die Programmierung eingef{\"u}hrt hat. Mit Freunden und
Kommilitonen habe ich w{\"a}hrend der Schul- und Studienzeit viel {\"u}ber die
Physik gesprochen, um ihnen ebenfalls Freude und Leidenschaft f{\"u}r dieses
Thema zu vermitteln.

%In diesem Anschreiben werde ich kurz die wichtigsten Punkte meines Studiums und die
%meiner bisherigen Berufserfahrung zusammenfassen, welche auf der Grundlagenforschung und der Entwicklung hochkomplizierter Software basieren.

An der Perugia Universit{\"a}t in Italien habe ich Physik studiert. Im
Anschluss daran habe ich in Deutschland an der Humboldt Universit{\"a}t zu
Berlin im Fachbereich Teilchen-/Beschleunigerphysik promoviert. Inhalt meiner
Studien waren die Durchf{\"u}hrung von Simulationen und Tests wichtiger
Komponenten von Teilchenbeschleunigern und Detektoren.

Seit dem erfolgreichen Abschluss meiner Promotion arbeite ich am
Forschungsinstitut DESY in Hamburg. Bei DESY engagiere ich mich mit komplexen
Problemen in der modernen Grundlagenforschung, wo eine l{\"o}sungsorientierte
Denkweise erfordert ist. Dar{\"u}ber hinaus habe ich meine Fachkenntnisse im
Bereich Programmierung erweitert, und besch{\"a}ftige mich mit professionellen
Methoden in der Softwareentwicklung. Neben Forschung und Entwicklung geh{\"o}rt
die Betreuung von Studierenden zu meinen Aufgaben, was mir viel Freude bereitet.


%Die Betreuung von Studierenden dort bereitete mir viel Freude. 

Mein Interesse an der Lehre ist mit den Jahren gewachsen. Deshalb absolvierte
ich im Februar 2015 eine 3-w{\"o}chige Hospitation im Schuldienst beim
Gymnasium Grootmoor zu absolvieren.


% Hier arbeitete
%ich f{\"u}r das ALFA-Projekt des ATLAS Experiments am CERN in Genf. Innerhalb des
%Projekts habe ich mich mit Detektorstrahltestmessung besch{\"a}ftigt. In diesem Projekt
%war ich der Koordinator der  "`Data Preparation"' Gruppe, die f{\"u}r die Aufbereitung
%von gro{\ss}en Mengen von Daten zust{\"a}ndig und an der physikalischen Datenanalyse
%beteiligt war. Au{\ss}erdem geh{\"o}rte ich zu dem Softwareentwicklungsteam des ATLAS
%Experiments.
%
%Im Juni 2012 wechselte ich zu der Strahlendetektor-Gruppe in DESY. In der Gruppe
%bin ich zust{\"a}ndig f{\"u}r die Entwicklung des Softwareframeworks f{\"u}r Datenaufbereitung,
%Daten- und Bildverarbeitung und der Software f{\"u}r das Data Acquisition System.
%Au{\ss}erdem beteilige ich mich an der Auswertung der Testdaten.

Aufgrund meiner oben genannten T{\"a}tigkeiten entsprechen meine Qualit{\"a}ten und
Kompetenzen sehr gut den Anforderungen einer T{\"a}tigkeit als Lehrer. 
%{\"U}ber eine Einladung zu einem pers{\"o}nlichen Vorstellungsgespr{\"a}ch freue ich mich sehr.


\closing{Mit freundlichem Gru\ss}
\enlargethispage{6\baselineskip}

\end{letter}
\end{document}

\endinput
