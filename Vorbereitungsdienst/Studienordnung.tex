\documentclass[a4paper,11pt]{article}

%\usepackage[latin1]{inputenc}
%\usepackage[italian]{babel}
%\usepackage{graphicx}
\usepackage{amssymb}
\usepackage{caption,booktabs,array}
\usepackage{setspace} \usepackage{geometry} \usepackage{hyperref}
\geometry{a4paper,tmargin=30mm,bmargin=30mm,lmargin=30mm,rmargin=30mm}

\begin{document}
\pagestyle{empty}
\begin{center}
  \huge{Studienordnung}
\end{center}
\vspace{1cm}

Im Folgenden wird die Studienordnung des Studiengangs in Physik der
Perugia Universit\"at zusammengefasst. Die Zusammenfassung beruht auf den Seiten 3 - 12 des
Dokumentes `Notiziario facolt\`a di scienze matematiche fisiche e naturali,
corso di laurea in Fisica''.\\

Die Dauer des Studiengangs in Physik betr\"agt 4 Jahre.\\

Der Studiengang ist in zwei Teilen organisiert. Die ersten drei Jahre bestehen
aus 14 Basiskursen. Jeder Kurs hat eine Dauer von zwei Semestern (ein
Jahr).\\

Sp\"atestens im zweiten Jahr m\"ussen Kenntnisse einer Fremdsprache (in der Regel
Englisch oder Franz\"osisch) nachgewiesen werden.\\

F\"ur das letzte Jahr ist die Auswahl einer Spezialisierung erforderlich und der
entsprechende Studienplan muss erstellt werden. Die Auswahl muss sp\"atestens im 
dritten Jahr erfolgen.\\

Die von mir ausgew\"ahlte Spezialisierung ``Atomare und Subatomare Physik''
sieht f\"ur das letzte Jahr 3 Zwei-Semester und 2 Ein-Semester Kurse vor.
\\

F\"ur den Unterricht besteht keine Anwesenheitspflicht. Nach dem Abschluss eines Kurses
ist eine Pr\"ufung \"uber den Inhalt des Kurses vorgesehen (in der Regel schriftlich  und  m\"undlich abzulegen).\\
%f\"ur die Absolvierung der Pr\"ufungen keine Zeitfrist.

Die erfolgreiche Absolvierung der Pr\"ufungen f\"ur die Kurse Mathematische Analyse I und II und
Allgemeine Physik I und II ist Voraussetzung f\"ur die Ablegung der Pr\"ufungen f\"ur die Kurse
des dritten und vierten Jahres.\\

Die Zulassung zur Verteidigung der Abschlussarbeit (Diplomarbeit) erfolgt f\"ur
den Studiengang mit Spezialisierung ``Atomare und Subatomare Physik'' nach der
erfolgreichen Absolvierung der Pr\"ufungen von 17 Zweisemesterkursen und 2
Einsemesterkursen.\\

Das Thema der Diplomarbeit muss der Spezialisierung entsprechen.\\

Es folgt eine Tabelle mit den abgelegten Pr\"ufungen (siehe auch das Dokument
Diplomzeugnis).

\newpage

\begin{table}[hbt]
%   \vspace*{-.5\baselineskip}
\centering
\begin{tabular}{p{5cm}cl}
\toprule
\textbf{Kurs} & \textbf{Dauer}                      &
\multicolumn{1}{c}{\textbf{Notiz}}
\\
\midrule
Mathematische Analyse I  & 2 Semester & Grundlage und
Differentialrechnung \\
Physikalische Versuche I & 2 Semester            & Experimentelle
Methoden in der Physik\\
Allgemeine Physik I     &  2 Semester           &   Newtonsche
Mechanik und Thermodynamik        \\
Geometrie       & 2 Semester           &   Algebra vom
Vektorraum \\
\midrule
Mathematische Analyse II  & 2 Semester & Reihen und Differentialgleichung \\
Allgemeine und Anorganische Chemie & 2 Semester & \\
Allgemeine Physik II & 2 Semester &  Elektromagnetismus\\
Physikalische Versuche II & 2 Semester & Elektrotechnik und Elektronik \\
Rationale Mechanik mit Elementen der Statistischen Mechanik & 2 Semester & \\
\midrule
Physikalische Versuche III & 2 Semester & Fortgeschrittene Themen\\
Atomare uns Subatomare Physik & 2 Semester & \\
Theoretische Physik & 2 Semester & Grundlage der Quantenmechanik\\
Mathematische Methoden f\"ur die Physik & 2 Semester &\\
Struktur der Materie & 2 Semester &\\
\midrule
Elementarteilchenphysik & 2 Semester & \\
Laboratorium f\"ur subatomare Physik & 2 Semester &\\
Experimentelle Methoden der (sub)atomaren Physik & 1 Semester & \\
Feldtheorie (II Module) & 1 Semester & Gravitation\\
Quantentheorie & 2 Semester & Einf\"uhrung in die Quantenfeldtheorie\\
\bottomrule
\end{tabular}
%\label{l2ea4-t1}
%   \vspace*{-\baselineskip}
\end{table}


\end{document}




 



