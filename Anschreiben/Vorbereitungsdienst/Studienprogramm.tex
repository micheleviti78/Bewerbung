\documentclass[a4paper,11pt]{article}

%\usepackage[latin1]{inputenc}
%\usepackage[italian]{babel}
%\usepackage{graphicx}
\usepackage{amssymb}
\usepackage{setspace} \usepackage{geometry} \usepackage{hyperref}
\geometry{a4paper,tmargin=30mm,bmargin=30mm,lmargin=30mm,rmargin=30mm}

\begin{document}
\pagestyle{empty}
\begin{center}
  \huge{Studienprogramm}
\end{center}
\vspace{1cm}

Die Liste der abgelegten und bestandenen Pr\"ufungen ist in dem Diplomzeugnis zu
erhalten das mit den Bewerbungsunterlagen verschickt wurde. Hier wird der Inahlt
jeder Prüfung laut dem Dokument ``Notiziario di Fisica'' in Stichpunkten
zusammengefasst. Die Seitennummern wo der Inahlt im Notiziario di Fisica zu
finden ist, wird neben dem Titel notiert.

{\bfseries Mathematische Analyse I, 13-16}
\begin{itemize}
  \item Zahlbereiche
  \item Kombinatorik
  \item Mengenlehre
  \item Folge, Grenzwert, Kontinuierliche Funktionen
  \item Differentialrechnung für Funktionen mit einer Variable
  \item Differentialrechnung für Funktionen mit mehreren Variablen
  \item Implizite Funktionen
  \item Riemannsches Integral
  \item Generalisiertes Integral
\end{itemize}

{\bfseries Mathematische Analyse II, 16-18}
\begin{itemize}
  \item Potenzreihe, Abelscher Grenzwertsatz, Fourierreihe
  \item Periodische Funktionen
  \item Cauchy-Problem und differentialgleichungen
  \item Partielle Ableitung
  \item Messbare Funktionen
  \item Kurve in $\mathbb{R}^n$
  \item Implizite Funktionen in 2 oder 3 Dimensionen
  \item Differentialform
  \item Der gaußsche Integralsatz und Stokesformel
\end{itemize}

{\bfseries Allgemeine und Anorganische Chemie 18-20}
\begin{itemize}
  \item Allgemeine Einführung
  \item Chemische Thermodynamik
  \item Phasengleichgewicht
  \item Chemisches Gleichgewicht
  \item Chemische Kinetik
  \item Atomistische Struktur
  \item Chemische Bindung und Molekülstrukturen
  \item Statistische Thermodynamik
\end{itemize}

{\bfseries Physikalische Versuche I 21-23}
\begin{itemize}
  \item Physikalische Gr\"o{\ss}e und Einheitensysteme
  \item Messfehler und Fehlerfortpflanzung
  \item Statistische Datenanalyse
  \item Messung von mechanischen und thermischen Gr\"{\ss}en
  \item Laborversuche
  \item Automatisierte Verarbeitung
\end{itemize}

{\bfseries Physikalische Versuche II 24-26}
\begin{itemize}
  \item Elektrische Leitfähigkeit
  \item Gleichstromschaltungen
  \item Methode und Ger\"ate f\"ur Gleichstrommessung
  \item Wechselstromschaltungen
  \item Methode und Ger\"ate f\"ur Wechselstrommessung
  \item Halbleiterdiode
  \item Bipolartransistor
  \item Feldeffekttransistor
\end{itemize}

{\bfseries Physikalische Versuche III 27-28}
\begin{itemize}
  \item Wahrscheinlichkeitsverteilungen
  \item Lineare und Nichtlineare Regression
  \item Computerarchitektur
  \item Betriebssysteme
  \item Programmiersprachen BASIC und FORTRAN
  \item Operationsverstärker
  \item Digitalelektronik
  \item Teilchen- und Lichdetektoren
  \item Vakuumpumpen
\end{itemize}

{\bfseries Allgemeine Physik I 28-32}
\begin{itemize}
  \item Einf\"urung. Definition von physikalischen Einheiten. Länge, Zeit und
  Masse. Internationales Einheitensystem, Basiseinheiten und abgeleitete
  Einheiten. 
\end{itemize}



\end{document}
