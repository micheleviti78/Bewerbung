\documentclass[ebner,paper=a4,fontsize=11pt,ngerman,BCOR=10mm]{scrlttr2}% 

\KOMAoptions{foldmarks=false,backaddress=false,parskip=full}

\usepackage[ngerman]{babel}
\usepackage[T1]{fontenc}     
\usepackage[latin9]{inputenc}

\usepackage{marvosym}    

\usepackage{url}

\firstfoot{}

\begin{document}
\pagestyle{empty}

\begin{letter}{European XFEL GmbH \\
Holzkoppel 4 \\
22869 Schenefeld \\
Germany}

\setkomavar{date}{\today}
\setkomavar{subject}{Application as Software Engineer} 

\setlength{\parindent}{15pt}

\opening{Dear Dr. Brockhauser,}

My name is Michele Viti and I would like to apply for the position as Programmer
- Scientific Instrument Support Engineer.\\
\indent I studied elementary particle physics at the Perugia university, with a
thesis on the simulation of the trigger for the experiment KOPIO on a rare decay of
the K-meson. I started my PhD in 2006 in DESY Zeuthen, working in the linear
collider group. My work concentrated on both simulations and beam test data
analysis for the development of systems for a precise beam energy measurement
at the International Linear Collider.\\
\indent After successfully defending my PhD thesis, I started my postdoctoral
research in DESY Hamburg. In DESY Hamburg I have been working for three groups, namely
the ATLAS group, the HASYLAB detector group and my current group, the machine
beam control (Maschine Strahlkontrollen, MSK) group.\\
In the ATLAS experiment I was part of the ALFA project leading the data
preparation group which was responsible for the preprocessing and preparation of
the beam test and tunnel data.  \\
In the HASYLAB detector group I worked for the PERCIVAL project,
developing the data acquisition system for the test setup, and a framework for the processing
and the analysis of the data. As well as software tasks, I contributed to
physical studies like the determination of the linearity of the ADCs and
measuring the resolution of the detector.\\ 
In the MSK group I'm currently developing the software for the slow
control and data acquisition for the beam arrival time monitors and the library for the
stepper motor control.\\
\indent Starting as a pure physicist, I moved through these years to more
engineering tasks. This was driven by both personal interest and work necessity. This
evolution matured in the current group where I have the chance to work
together with pure programmers and software engineers.\\
\indent What I can offer for this position is an excellent combination of very
good physics knowledge and professional experience in the software development,
especially for slow control, data acquisition and analysis for measuring
instruments and photon detectors.

\closing{with best regards}
\enlargethispage{6\baselineskip}

\end{letter}
\end{document}

\endinput
