\documentclass[ebner,paper=a4,fontsize=11pt,ngerman,BCOR=10mm]{scrlttr2}% 

\KOMAoptions{foldmarks=false,backaddress=false,parskip=full}

\usepackage[ngerman]{babel}
\usepackage[T1]{fontenc}     
\usepackage[latin9]{inputenc}

\usepackage{marvosym}    

\usepackage{url}

\firstfoot{}

\begin{document}
\pagestyle{empty}

\begin{letter}{Ecker + Ecker GmbH\\
Frau Dr. P\"utz\\
Mittelweg 144\\
20148 Hamburg }

\setkomavar{date}{\today}
\setkomavar{subject}{Bewerbung als Physiker} 

\opening{Sehr geehrte Frau Dr. P\"utz,} 

mein Name ist Michele Viti. \"Uber die Webseite der Joebb\"orse habe
ich das Stellengebot der Firma Ecker+Ecker mit gro{\ss}em Interesse
gelesen und m�chte mich darauf bewerben.  W\"ahrend meiner
Berufserfahrung habe ich mich mit Datenprozessierung, Datenauswertung
und der entsprechenden Softwareentwicklung besch�ftigt.  Au{\ss}erdem
habe ich an zahlreichen Meetings, Workshops und internationalen
Konferenzen teilgenommen bei denen ich Daten und Resultate
pr\"asentiert habe. In diesem Anschreiben werde ich kurz die wichtigen
Punkte meines Studiums und meiner Berufserfahrung zusammenfassen.

%Software entwicklung real time system expertiment

%\"Uber die StepStone Webseite habe ich von Ihrem Stellenangebot
%erfahren und m\"ochte mich darauf bewerben.

An der Perugia Universit\"at in Italien habe ich Physik studiert. Im
Anschluss daran habe ich in Deutschland an der Humboldt Universit\"at
zu Berlin im Fachbereich Teilchen-/Beschleunigerphysik promoviert. In
meinen Studien handelte es sich um Simulation, Test wichtiger
Komponenten von Teilchenbeschleunigern und Detektoren.

Nach dem erfolgreichen Abschluss meiner Promotion habe ich eine Stelle
als Post-Doktorand am Forschungsinstitut DESY in Hamburg
angenommen. Hier arbeite ich f\"ur das ALFA-Projekt des ATLAS
Experiments am CERN in Genf. Innerhalb dieses Projekts habe ich mich
mit Detektorstrahltestmessung besch\"aftigt. Zurzeit bin ich der
Koordinator der "`Data Preparation"' Gruppe, die f\"ur die Auswertung
der Daten zust\"andig und an der physikalischen Datenanalyse beteiligt
ist. Au{\ss}erdem geh�re ich zu dem Softwareentwicklungsteam des ATLAS
Experiments.

Meine Beitr\"age zu diesen Studien umfassen einen breiten Bereich,
insbesondere Computersimulation, Datenaufnahme und Datenauswertung. Um
diese Aufgaben zu erf�llen, habe ich erweiterte statistische
Analyseverfahren verwendet, fundierte Programmierkenntnisse,
insbesondere objektorientierte Programmierung, erworben, komplexe
Software Frameworks benutzt und Algorithmen erzeugt.

%%f\"ur die Entwicklung neuer
%%Komponenten und Detektoren, Planung und Teilnahme an Tests f\"ur
%%Protoypen und Datenanalysen. Details \"uber diese Aufgaben k\"onnen
%%meinem Lebenslauf entnommen werden.

Ich habe eine hohe Flexibilit\"at, Selbstst\"andigkeit und
Belastbarkeit entwickelt, um mich erfolgreich verschiedenen Themen und
neuen Herausforderungen zu stellen.Seit mehr als 6 Jahren arbeite ich
innerhalb internationaler Kollaborationen mit verschiedenen
weltber\"uhmten Instituten und Universit�ten, wie z.B. der Stanford
Universit�t in der USA, dem Deutsches Elektronen Synchrotron (DESY) in
Deutschland, dem Forschungszentrum CERN in Genf und dem Budka Institut
in Novosibirsk, in denen sehr gute Kommunikationsf\"ahigkeiten in
einer multikulturellen Umgebung und teamorientierte Arbeit erfordert
sind.

%Seit ca. 8 Jahren arbeite ich in
%gro{\ss}en internationalen Kollaborationen, .

Aufgrund meiner oben genannten T\"atigkeiten entsprechen meine
Qualit\"aten und Kompetenzen sehr gut den Anforderungen dieser Stelle
als Physiker. \"Uber eine Einladung zu einem pers\"onlichen
Vorstellungsgespr\"ach w\"urde ich mich sehr freuen.

\closing{Mit freundlichem Gru\ss}
\enlargethispage{6\baselineskip}

\end{letter}
\end{document}

\endinput
